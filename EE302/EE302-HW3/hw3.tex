\documentclass[11pt]{article}
\usepackage{graphicx}
\usepackage[export]{adjustbox}
\usepackage{float}
\usepackage{listings}
\usepackage{amsmath}
\usepackage{color}
\definecolor{lightgray}{gray}{0.9}
\title{EE302 Homework 1}
\date{2018\\ March}
\author{Nail Tosun - 2094563 -Section 5\\ Electric and Electronic Engineering Departmant, METU}
\begin{document}
\maketitle
1)

a)
\begin{table}[H]
\centering
\caption{Routh-Hurwitz Table for part-a}
\label{my-label}
\begin{tabular}{llll}
$s^3$ & \multicolumn{1}{c}{1}  & \multicolumn{1}{c}{10}  & \multicolumn{1}{c}{0} \\
$s^2$ & \multicolumn{1}{c}{20} & \multicolumn{1}{c}{400} & \multicolumn{1}{c}{0} \\
$s$   & -10                    & 0                       & 0                     \\
1     & 400                    & 0                       &                      
\end{tabular}
\end{table}
There are two positive root therefore system is unstable.
b)

\begin{table}[H]
\centering
\caption{Routh-Hurwitz Table for part-b}
\label{my-label}
\begin{tabular}{llll}
$s^5$ & \multicolumn{1}{c}{1} & \multicolumn{1}{c}{2} & \multicolumn{1}{c}{3} \\
$s^4$ & \multicolumn{1}{c}{3} & \multicolumn{1}{c}{6} & \multicolumn{1}{c}{1} \\
$s^3$   & $\epsilon=0.01$                 & 2.67                  & 0                     \\
$s^2$     & -794                  & 1                     & 0                     \\
$s$   & 2.67                  & 1                     & 0                     \\
1  & 1                     & 0                     & 0                    
\end{tabular}
\end{table}
There are two positive root therefore system is unstable.

c)
\begin{table}[H]
\centering
\caption{Routh-Hurwitz Table for part-c}
\label{my-label}
\begin{tabular}{llll}
$s^4$ & \multicolumn{1}{c}{1}  & \multicolumn{1}{c}{2}  & \multicolumn{1}{c}{-8} \\
$s^3$ & \multicolumn{1}{c}{-1} & \multicolumn{1}{c}{-4} & \multicolumn{1}{c}{0}  \\
$s^2$ & -2                     & -4                     & 0                      \\
$s$   & 4                      & 0                      &                        \\
1     & -8                     &                        &                       
\end{tabular}
\end{table}
There are three positive root therefore system is unstable.
\begin{table}[H]
\centering
\caption{Routh-Hurwitz Table for part-d}
\label{my}
\begin{tabular}{llll}
$s^5$ & \multicolumn{1}{c}{1} & \multicolumn{1}{c}{16} & \multicolumn{1}{c}{100} \\
$s^4$ & \multicolumn{1}{c}{2} & \multicolumn{1}{c}{32} & \multicolumn{1}{c}{200} \\
$s^3$ & 0                     & 0                      & 0                       \\
$s^2$ &                       &                        &                         \\
$s$   &                       &                        &                         \\
1     &                       &                        &                        
\end{tabular}
\end{table}
Entire row is zero. Therefore i take derivative of upward equation and added its coefficents to zero row.
\[\frac{d}{ds}(2s^4+32s^2+200)\]
\[8s^3+64s\]
then new table become;
\begin{table}[H]
\centering
\caption{Routh-Hurwitz Table for part-d ver2}
\label{my-lael}
\begin{tabular}{llll}
$s^5$ & \multicolumn{1}{c}{1} & \multicolumn{1}{c}{16} & \multicolumn{1}{c}{100} \\
$s^4$ & \multicolumn{1}{c}{1} & \multicolumn{1}{c}{8}  & \multicolumn{1}{c}{0}   \\
$s^3$ & 8                     & 100                    & 0                       \\
$s^2$ & -4.5                  & 0                      & 0                       \\
$s$   & 100                   & 0                      & 0                       \\
1     & 0                     &                        &                        
\end{tabular}
\end{table}

4)
Matlab script to finding roots;



\lstset{
    showstringspaces=false,
    basicstyle=\ttfamily,
    keywordstyle=\color{blue},
    commentstyle=\color[grey]{0.6},
    stringstyle=\color[RGB]{255,150,75}
}

\newcommand{\inlinecode}[2]{\colorbox{lightgray}{\lstinline[language=#1]$#2$}}

\inlinecode{java}{
p1 = [1 20 10 400];}

\inlinecode{java}{
p2 = [1 3 2 6 3 1];}

\inlinecode{java}{
p3 = [1 -1 2 -4 -8];}

\inlinecode{java}{
p4 = [1 2 16 32 100 200];}

\inlinecode{java}{
roots1 = roots(p1);}

\inlinecode{java}{
roots2 = roots(p2);}

\inlinecode{java}{
roots3 = roots(p3);}

\inlinecode{java}{
roots4 = roots(p4);}
\vspace{5mm} %5mm vertical space

Results are (in 2 significant digit);
polynomial 1 has roots at 

$\lambda_1= -20,47 + 0,00i$

$\lambda_2= 0,23 + 4,41i$

$\lambda_3= 0,23 - 4,41i$

\vspace{5mm}

polynomial 2 has roots at

$\lambda_1=-2,91$

$\lambda_2=0,23 + 1,33i$

$\lambda_3=0,23 - 1,33i$

$\lambda_4=-0,28 + 0,34i$

$\lambda_5=-0,28 - 0,34i$

\vspace{5mm}

polynomial 3 has roots at

$\lambda_1=2.00$

$\lambda_2=2.00i$

$\lambda_3=-2.00i$

$\lambda_4=-1.00$

\vspace{5mm}

polynomial 4 has roots at

$\lambda_1=1,00 + 3,00i$

$\lambda_2=1,00 - 3,00i$

$\lambda_3=-1,00 + 3,00i$

$\lambda_4=-1,00 - 3,00i$

$\lambda_5=-2,00$
\end{document}
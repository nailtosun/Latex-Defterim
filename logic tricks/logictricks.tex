\documentclass[11pt]{article}
\usepackage{graphicx}
\usepackage[export]{adjustbox}
\usepackage{float}
\usepackage{amsmath}
\usepackage{siunitx}
\title{Logic Tricks}
\date{2018\\ March}
\author{Nail Tosun \\ Electric and Electronic Engineering Departmant, METU}
\begin{document}
\maketitle
\section*{Binary Numbers}
\subsection*{r's complement}
General formula is following;
\[r^n-N\]
r is radix n is integer side length and N is the number that you complement.
\subsubsection*{Shortcut}
Starting from the right to left;

Leave all zeros no change;

After first zero r-$digit_{afterzero}$

Then (r-1) - all digits.

\subsection*{r-1 complement}
General formula 
\[r^n-r^{-m}-N\]
\subsubsection*{Shortcut}
Subtract all the digits with r-1
\subsection*{2's complement}
\subsubsection*{Shortcut}
Switch all the bits and add 1 or

Keep the same with all zero (right to left). After first non-zero term (don't change first non-zero term), toggle each bit.
\subsection*{1's complement}
\subsubsection*{Shortcut}
Just switch all the bits.


\subsection*{Signed magnitude convention}
\subsubsection*{Signed Magnitude}
First leftmost bit is sign bit other registers indicate magnitude.
\subsubsection*{Signed 1's complement}
First leftmost bit is sign bit.
If first leftmost bit is zero (indicates positive)

Other bits represents +number

If first leftmost bit is one (indicates negative)

Put a minus sign

Takes 1's complement.

\subsubsection*{Signed 2's complement}
First leftmost bit is sign bit.
If first leftmost bit is zero (indicates positive)

Other bits represents +number

If first leftmost bit is one (indicates negative)

Put a minus sign

Takes 2's complement.
There is one representation for zero.

\subsubsection*{Boundaries}
3 bit Signed 1's complement and magnitude methods
\[-3\leq X \leq 3\] 
2's complement method has extra one number(due to only has one zero.)
\[-4 \leq X \leq 3\]

\subsection*{Arithmetic addition in 2's complement}
Take both number 2's complement.

Sum it up. 

Fuck the carry bit there is no information here. Right of the carry bit is magnitude bit. Put your sign if it is 1 (indicates negative) take 2's complement to find its absolute value.

\subsection*{Overflow}
Sign bits of the both operand are same and result sign is different. It indicates overflow.

Overflow can happen if both operands sign are same.

\subsection*{Extension of the bits}

Copy the sign bit and paste left bits. 
001 's extension the 6 bit is 000001;
100 's extension to 6 bit is  111100;
\subsection*{Excess 3}
BCD+3

Self complementing

\subsection*{Parities}
Finding the error in data transfer
\subsubsection*{Odd parity}
Add one bit to ensure sum of all bits odd
\subsubsection*{Even parity}
Add one bit to ensure sum of all bits even

\section*{Substraction with complentary}
\subsection*{Substraction with r's complement}
Operation of $(M)_r-(N)_r$ where M and N are unsigned integer.

Write N's r's complement and add with M;
\[M+r^n-N\]
if $M>N$ sum is produces carry bit. Ignore the carry bit. Sum is your result.

if $N>M$ sum is \textbf{don't produce carry bit} so take r's complement of the result.

Don't forget to put minus sign.
\subsection*{Substraction with r-1 complement}
Take the cıkarın operand r-1 complement.

Sum it up 

If there is a \textbf{carry out} this indicates number is \textbf{positive}. add carry out to least significant bit. 

If there is \textbf{no carry out}. It means the number is negative. Take ones complement
\end{document}